\documentclass{article}

%------------------------------------------------
% Imports
%------------------------------------------------
\usepackage{amsmath}
\usepackage{amssymb}
\usepackage{geometry}
\usepackage{graphicx}
\usepackage{etoolbox} % block quotes
\usepackage{setspace} % linespacing
\usepackage[colorlinks,allcolors=red]{hyperref} % citation hyperlinks

%------------------------------------------------
% Settings
%------------------------------------------------
\onehalfspacing
\AtBeginEnvironment{quote}{\singlespacing\small}

%------------------------------------------------
% Commands
%------------------------------------------------
\newcommand{\tx}[1]{\text{#1}}
\newcommand{\ti}[1]{\textit{#1}}
\newcommand{\p}{^\prime}

% citations
\renewcommand{\cite}[1]{\hyperlink{#1}{#1}}
\newcommand{\reference}[3]{\bibitem{#1} \hypertarget{#1}{} #1. \ti{#2}}

% annotated bibliography
\newcommand{\annbibtitle}[2]{\subsection*{#1. \ti{#2}}}

\begin{document} % +===+===+===+===+===+===+===+

%------------------------------------------------
% Title
%------------------------------------------------
\begin{center}
	\huge{\bf Dependent Autonomy} % paper title
    \\[0.75cm] 
	\large{Henry Blanchette} % paper author
    \\[0.5cm]
	\large{Metaphysics of Science \\ Reed College} % paper organization
    \\[1.0cm]
\end{center}

%------------------------------------------------
% Abstract
%------------------------------------------------
\begin{abstract}
	This is the abstract.
\end{abstract}

%------------------------------------------------
% Introduction
%------------------------------------------------
\section{Introduction}

This is the introduction.

% Citation example:
A thought that a previous philosopher wrote about (\cite{Wilson J. 2018}).

%------------------------------------------------
% Bibliography
%------------------------------------------------
% TODO: decide if newpage or not
\newpage
% TODO: order once all the entries are in
\begin{thebibliography}{9}

\reference{Wilson J. 2018}{Metaphysical Emergence}
    {Department of Philosophy at University of Toronto, ON CA}

\reference{Fodor J. A. 1974}{Special Sciences}
    {\ti{Synthese}, Vol. 28, No. 2 (Oct., 1974), pp. 97-115. Springer}

\reference{Bedau M. 2018}{Downward Causation and the Autonomy of Weak Emergence}
    {Department of Philosophy at Reed College, OR USA}

\reference{Bedau M. 2008}{Is Weak Emergence Just in the Mind?}
    {Minds and Machines 18:443–459. Springer}

\reference{Wolfram S. 1985}{Undecidability and Intractability in Theoretical Physics}
    {Physical Review Letters, Volume 54 Number 8. The American Physical Society}

\reference{McLaughin B. P. 1992}{The Rise and Fall of British Emergentism}
    {Reprinted in M. A. Bedau \& P. Humphreys (Eds.). (2008). \ti{Emergence: Contemporary readings in philosophy and science} (pp. 19-60). Cambridge: MIT Press}

\reference{McLaughin B. P. 1997}{Emergence and Supervenience}
    {Reprinted in M. A. Bedau \& P. Humphreys (Eds.). (2008). \ti{Emergence: Contemporary readings in philosophy and science} (pp. 81-98). Cambridge: MIT Press}

\reference{Hempel C., Oppenheim P. 1965}{On the Idea of Emergence}
    {Reprinted in M. A. Bedau \& P. Humphreys (Eds.). (2008). \ti{Emergence: Contemporary readings in philosophy and science} (pp. 61-68). Cambridge: MIT Press}

\reference{Searle J. 1992}{Reductionism and the Irredicibility of Conciousness}
    {Reprinted in M. A. Bedau \& P. Humphreys (Eds.). (2008). \ti{Emergence: Contemporary readings in philosophy and science} (pp. 61-68). Cambridge: MIT Press}

\reference{Dennett D. C. 1991}{Real Patterns}
    {Reprinted in M. A. Bedau \& P. Humphreys (Eds.). (2008). \ti{Emergence: Contemporary readings in philosophy and science} (pp. 189-206). Cambridge: MIT Press}

\reference{Humphreys P. 1997}{How Properties Emerge}
    {Reprinted in M. A. Bedau \& P. Humphreys (Eds.). (2008). \ti{Emergence: Contemporary readings in philosophy and science} (pp. 111-126). Cambridge: MIT Press}

\end{thebibliography}

%------------------------------------------------
% Annotated Bibliography
%------------------------------------------------
\newpage
\section*{Annotated Bibliography}

% TODO
\annbibtitle{Wilson J. 2018}{Metaphysical Emergence}

\subsubsection*{Chapters 1 \& 2}
\subsubsection*{Chapters 3 \& 4}
\subsubsection*{Chapter 5}

\annbibtitle{Fodor J. A. 1974}{Special Sciences}

Consider and contrast two scientifically fundamental perspectives: the \ti{Unity of Science} and the \ti{Generality of Physics}. The Unity claim is actually stronger than the Generality claim; Unity of Science requires all sciences to have as their aim the construction of a physics-termed explanation of their studied phenomena, while Generality of Physics requires only that instances of studied phenomena be completely explainable in terms of physics. The Unity claim is, in fact, the central claim of Redictive Physicalism.

However, there is a seeming paradox in the Unity claim - it pronounces that the continued success of the special sciences is just more and more evidence that they ought to be discontinued; they stray further and further from an explanation in terms of physics. The \ti{special sciences} are sciences that do not deal directly or appeal to physical explanations.

\ti{Reductive Physicalism} is the view that all special sciences must reduce to physics. For some special-science relationship $S_1 \rightarrow S_2$, reductivism requires there be a reduction with to physical predicates $P_1, P_2$ such that

$$
\begin{array}{rcll}
    S_1 x & \rightarrow & S_2 x & \tx{(1)} \\
    S_1 x & \rightleftharpoons & P_1 x & \tx{(2a)} \\
    S_2 x & \rightleftharpoons & P_2 x & \tx{(2b)} \\
    P_1 x & \rightarrow & P_2 x & \tx{(3)}
\end{array}
$$

This setup forms a sort of bridge between the special-science relationship and a physical relationship. Though, there are some problems with the existence of bridges like this and how it interacts with the meaning of $\rightarrow$.

A way to address the bridge problems is with \ti{Token Physicalism}, where all events that the special sciences talk about are physical events.
\begin{itemize}
    \item Token physicalism is weaker than materialism, which claims token physicalism and that every event falls under the laws of some science or other.
    \item Token physicalism is weaker than type physicalism, where every property mentioned in the laws of any science is a physical property. This in fact implies token physicalism.
    \item Token physicalism is weaker than reductive physicalism, which claims token physicalism and that there are no natural kind predicates in an ideally completed physics which correspond ot each natural kind predicate in any ideally completed special science.
\end{itemize}

Every science implies a taxonomy of the events in its univers eof discourse. It creates theoretically- and empirically- inspired vocabulary, which fall under the laws of the science by virtue of satisfying those predicates. Not every theoretical predicate is valid or good though.

To fix reductive physicalism, can allow bridge relations to be in the form

$$
    S_x \leftrightharpoons P_1 x \lor \cdots \lor P_n x
$$

where $P_1 x \lor P_n x$ is not the kind of natural kind predicate in the reducing science. This allows for ``bridge laws'' to not really be laws, since they are not law-like. Reducing with this kind of bridge law looks like

$$
    P_1 x \lor \cdots \lor P_n x \leftrightharpoons P\p_1 \lor \cdots P\p_m
$$

Thesis:

\begin{quote}
There are special sciences not because of the nature of our epistemic relation to the world, but because of the way the world is put together: not all natural kinds (not all the classes of things and events about which there are important, counterfactual supporting generalizations to make) are, or correspond to, physical natural kinds.

A way of stating the classical reductionism view: things which belong to different physical kinds ipso facto can have no projectable descriptions in common; that if x and y differ in those descriptions by virtue of which they fall under the proper laws of physics, they must differ in those descriptions by virtue of which they fall under any laws at all.

If science is to be unified, then all such taxonomies must apply to the same things. If physics is to be basic science, then each of these things had better be a physical thing. But it is not further required that the taxonomies which the special sciences employ themselves reduce to the taxonomy of physics. It is not required, and it is probably not true.
\end{quote}

\annbibtitle{Bedau M. 2018}{Downward Causation and the Autonomy of Weak Emergence}

In this paper, Bedau explores a foundation of Weak emergence as \ti{derivability only by simulation} (or \ti{underivable except by simulation}).

The core precepts for the concept of emergence are:
\begin{enumerate}
    \item Emergent phenomena are dependent (ontologically) on underlying processes.
    \item Emergent phenomena are autonomous (causally) from underlying processes.
\end{enumerate}

All concepts of emergent base on the concept of the emergence property. The out-springing concepts include emergent entities (which are entities with emergent properties), emergent powers (which are powers bestowed by emergent properties), etc.

A property $P$ is underivable except by simulation if and only if $P$ can only be derived from a system, even a system causally closed at the lowest level, through a complete simulation of the low-level causes (e.g. \ti{gliders} in the Game of Life). Bedau poses that $P$ is Weakly emergent.

Basically, the idea is the if a property is interesting and complex in a formal way, in that it unpredictable arises from the low-level properties of a system, then it is indeed emergent (and, specifically, Weakly emergent).

It is important to note that this is a metaphysical point about emergent properties which is independent from epistemological concerns. For example, just because we haven't yet found a derivation sans simulation for a property doesn't mean that it is emergent - it just \ti{looks} emergent. Of course, a proof that there is not derivation is good, but such proofs are extremely hard to come by (especially because of the open $P =^? NP$ problem.)

Truely emergent properties are distinct from another kind of property that arises in similar circumstances - so-called \ti{resultant properties}. A property $P$ is a resultant property if and only if it is predictable from the properties of its parts. Often times, $P$ is just of a whole in and of itself and not an interesting emergence. $P$ of the whole that merely holds because of its being a whole as defined. For example, ``being a circle'' is a property that applies to a collection of points, but no single point in that collection shares the property of ``being a circle''. However, this is uninteresting and is directly predictable from the definition of a circle and the properties of points, without having to do any sort of scaling computation to figure it out.


In the \ti{Game of Life}, there are many good examples of properties that are underivable except by simulation. For example, the property ``finitely-expanding''. A worldstate that, after a finite number of time steps, reaches a maximum $m \times n$ area in the world such that no matter how many more time steps are run no cell will be born outside the area.

There are examples of non-finitely-expanding configurations, such as the famous \ti{Gospel Glider Gun} that creates gliders in intervals and sends them out into infinity. And there are examples of finitely-expanding configurations, such as any that consists of only non-interacting \ti{still-lifes}.

However, to derive if a worldstate is finitely-expanding, one needs to run a simulation. There is no general mathematical theorem that decides without doing the equivalent of a brute simulation. Even if a mega-computer could do the simulation instantaneous, it would still count as needing a simulation and thus ``finitely-expanding'' is an emergent property.

If, however, someone came up with a shortcut, it would prove that ``finitely-expanding'' is not emergent.

The main problem for emergence is that of \ti{downward causation} i.e. \ti{overdetermination}. If $S$ emerges from $P$ but $P$ is synchronically and ontologically dependent on $P$, then if $S$ causes a $P^*$ then surely we can say that $P$ causes $P^*$ as well. But if the low level of the $P$s is causally closed, then why do we need to talk about $S$ causing anything at all? It seems irrelevant and problematic in an overdeterminative way. But if $S$ was causally inefficacious, then it loses its status of autonomy from $P$ and is not really emergent in the first place. This is Kim's overdetermination argument.

But Kim's concerns are ill-founded. The idea of Weak emergence avoids the problems by yielding overdetermination as unproblematic, since it is causally reductionsitic and uses ontological dependence. But such weak emergence is important because it clearly defines autonomous and relevant structures.

Additionally, Weak emergence is not synchronic. It introduces a diachronic perspective on causation, where, for example, in the Game of Life for a macrostate to manifest an emergent property many time steps need to pass. For a microstate to have a property, it must be manifested in one or two time steps. In this case, emergent properties are causally relavant and distinct from lower-level properties.

Conclusion:
\begin{quote}
The advent of modern philosophy is conventionally presented as the Cartesian triumph over Aristotelian scholasticism. An Aristotelian thesis that attributed nature on the basis of a rich dependence on generating context was supplanted by the Cartesian antithesis that attributed reductionistic essences independent of context. Computer simulations allow weak emergence to extend reductionism into new territory, but they do so by embodying the idea that something's nature can depend on its genesis. Thus, the macro can depend on the context-sensitive process from which it arises and by which it is maintained. In this way, weak emergence can be viewed as a new synthesis.
\end{quote}

% TODO
\annbibtitle{Bedau M. 2008}{Is Weak Emergence Just in the Mind?}

% TODO
\annbibtitle{Wolfram S. 1985}{Undecidability and Intractability in Theoretical Physics}

% TODO
\annbibtitle{McLaughin B. P. 1992}{The Rise and Fall of British Emergentism}

% TODO
\annbibtitle{McLaughin B. P. 1997}{Emergence and Supervenience}

\annbibtitle{Hempel C., Oppenheim P. 1965}{On the Idea of Emergence}

The key observation is that people attribute to emergence much more than they should. For example, emergent phenomenon are commonly thought of as

\begin{itemize}
    \item unpredictable
    \item mysterious
    \item a manifestation of newness
\end{itemize}

But, upon close analysis, this paper concludes that these intuitions about emergence would leave emergence as an empty concept - nothing could possibly satisfy it. Even the classic examples of emergent phenomenon, like the wetness of water, are still predictable given the properties of their parts.

Rather than thinking of emergence as an ontological even, think of it as merely epistemic. Consider psychology. Our brain is made of chemicals and we do study the chemistry involved in the processes of psychology, but we do not have an explanation of high-level psychological concepts in terms of the low-level chemistry (or physics, for that matter). So, the psychological phenomena are emergent from the chemistry, but they are not metaphysically new or unpredictable. We just, with our present theories and understanding, have not explained the high level in terms of the low level. So, as an epistemic condition, psychological properties are emergent. But when we do figure such an explanation, psychology will no longer be emergent.

% TODO
\annbibtitle{Searle J. 1992}{Reductionism and the Intractability of Consciousness}

\annbibtitle{Dennett D. C. 1991}{Real Patterns}

Consider the question ``are beliefs real?'' Is it correct to \ti{believe} in centers of gravity? Why or why not? There are examples of people that argue these kinds of beliefs are obviously unreal and others that argue that they are obviously real (e.g. Dretske).

What about the case of a completely arbitrary $x$-center, such as the center of Dennett's lost socks. Are such concepts and beliefs about just as real as with other, more ``legitimate'' $x$-centers? Or does deciding the reality of concepts and beliefs about just to do with usefulness or interestingness?

A question that naturally arises is ``should we treat mental states/patterns (e.g. belief) as \ti{real} in the same way and to the same degree as other patterns, such as electrons?''

Dennet gives the canonical \ti{Bar Code} example. It is unclear that the underlying Bar Code pattern is there for all of the cases, except for $E$ and $F$ where it starts to get iffy. In fact, in $F$, it is technically indistinguishable from random noise, from strictly a retrospective perspective (without looking at the generating code).

Dennett refers to Chaitin's idea of \ti{incompressibility}, in that patterns are recognizable because they are compressible while noise is incompressible. But compressibility comes with two degrees of freedom: accuracy and simplicity. Usually, it is a trade-off from one to the other. It is merely a design choice for which is more important in which circumstance, and that it isn't an inherent metaphysical fact about the reality of the concerned patterns themselves. There are also good examples of this trade-off in the \ti{Game of Life}.

A final thought:
\begin{quote}
Fine tuning could of course reduce these probabilities (of modeling which method was used to create a bar code pattern), but that is not my point. My point is that even if the evidence is substantial that the discernible pattern is produced by one process rather than another, it can be rational to ignore these differences and use the simplest pattern description (e.g. bar code) as one's way of organizing the data \dots
\end{quote}

\annbibtitle{Humphreys P. 1997}{How Properties Emerge}

The exclusion argument:

\begin{enumerate}
    \item If every event $x$ is causally sufficient for every other event $y$, then there is not $x\p$ distinct from $x$ that is causally relevant to $y$. (exclusion)
    \item For every physical event $y$, some physical event $x$ is casually sufficient for $y$. (physical determinism)
    \item For every physical event $x$ and mental event $x\p$, $x$ is distinct from $x\p$. (dualism)
    \item For every physical event $y$, no mental event $x\p$ is causally relevant to $y$. (conclusion)
\end{enumerate}

This is bad for dualist theories, including ones incorporating supervenience. However, this argument is invalid.

Say that $x$ is \ti{causally connected} to $z$ if and only if $x$ causes $z$ or visa versa. Then a fixed formulation of the same idea is:

\begin{enumerate}
    \item[$1\p$] If an event $x$ is causally sufficient for an event $y$, then no event $x\p$ distinct from and causally disconnected from $x$ is causally relevant to $y$. (exclusion)
    \item[$2\p$] For every $I$-level event $y$, some $I$-level event $x$ is causally sufficient for $y$. ($I$-determinism)
\end{enumerate}

But what about physical bio-determinism? It doesn't seem right because the first biological events must have been caused from some events from a non-biological level. So, this more general formulation $2\p$ doesn't work, and we should restrict to the 0-level.

\begin{enumerate}
    \item[$2\p$] For every 0-level event $y$, some 0-level event $x$ is causally sufficient for $y$. (0-determinism)
    \item[$3\p$] For every 0-level event $x$ and every $i$-level event $x\p_i$, where $i > 0$, $x$ is distinct from $x\p_i$. (pluralism)
\end{enumerate}

The it immediately follows that we can write

\begin{enumerate}
    \item[$\p$] For every 0-level event $y$, no $i$-level event $x\p_i$ that is causally disconnected from every 0-level event antecedent to $y$, where $i > 0$, is causally relevant to $y$. (new conclusion)
\end{enumerate}

From this line of revised argumentation, we can pose a new definition of ``emergent property'': A property is emergent if and only if it has novel causal powers from its object's parts.

\end{document} % +===+===+===+===+===+===+===+
